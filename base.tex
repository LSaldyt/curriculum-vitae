\documentclass[letterpaper,11pt]{article}
\usepackage[left=0.5in, right=0.5in, top=0.5in, bottom=0.5in]{geometry}
\usepackage{helvet}
\usepackage{setspace}
\usepackage[parfill]{parskip}
\usepackage[outline]{contour}
\contourlength{0.2pt}
\contournumber{10}
\renewcommand{\textbf}[1]{{\contour{black}{#1}}}

\renewcommand{\baselinestretch}{0.6} 
\renewcommand{\familydefault}{\sfdefault}
\setlength\itemsep{1em}
% Copied from old resume. TODO: Revise
\newcommand{\resitem}[1]{\item #1 \vspace{-3pt}}
\newcommand{\resheading}[1]{
    {\large \textbf{#1}}
    \sectionline
}
\newcommand{\colfill}{@{\extracolsep{\fill}}}
\newcommand{\ressubheading}[4]{
\begin{tabular*}{6.5in}{l\colfill r}
        \textbf{#1} & #2 \\
		\textit{#3} & \textit{#4} \\
\end{tabular*}\vspace{-6pt}}

% Section!
\newcommand{\ressection}[2]
{
\vspace{0.1in}
\begin{minipage}[t]{0.05\textwidth}
    \textbf{#1}
\end{minipage}
\hspace{0.05\textwidth}
\begin{minipage}[t]{0.9\textwidth}
    #2
\end{minipage}
}


\begin{document}
\begin{center}
    \huge Lucas Saldyt
\end{center}
\vspace{-0.2\baselineskip}
\begin{center}
    lucassaldyt@gmail.com $\bullet$ 505-506-1245 $\bullet$ https://github.com/LSaldyt
\end{center}
\vspace{-1.0\baselineskip}
\rule{1.0\textwidth}{1.7pt}
\vspace{-1.0\baselineskip}

\ressection{Education}{
    \ressubheading{Arizona State University}{Tempe, Arizona}{PhD in Computer Science, GPA: 4.0}{Sep. 2021 - Present}
    \ressubheading{Arizona State University: Barrett, The Honors College}{Tempe, Arizona}{Bachelor of Science in Computer Science, GPA: 3.71}{Sep. 2017 - May 2021}
    }
    \vspace{1.0\baselineskip}

%     \ressubheading{PathAI}{Boston, Massachusetts}{Software Intern (Machine Learning Deployment)}{May. 2020 - Aug. 2020}
%     \vspace{-0.5\baselineskip}
%  	\begin{itemize}
%  	    \resitem{Deployed state of the art machine learning models for cancer diagnosis and treatment in a profitable and robust real-world medical device}{(Python, tensorflow)}
%  	    \resitem{Converted web-based components to more efficient local programs, i.e. by writing a controller which overcame inefficiencies of Kubernetes}{(Minio, multiprocessing)}
%  	    \resitem{Networked with different departments and successfully integrated software components in Rust, Javascript, and Python}{(AWS, S3, Docker)}
%  	\end{itemize}
 	
\ressection{Experience}{
    \ressubheading{NASA Glenn Research Center}{Cleveland, Ohio}{Machine Learning Intern}{Jan. 2020 - May 2020}
    \vspace{-0.5\baselineskip}
 	\begin{itemize}
 		\resitem{Architected a modular data and machine learning pipeline which aggregates and refines image, article, and taxonomy data on 1.9 million living species}{(Python, neo4j)}
    \resitem{Experimented with EfficientNet CNN for classifying a large subset of biological species at 82\% top-1 accuracy efficiently}{(pytorch)}
 	    \resitem{Created a custom search engine based on original Google publications}{}
 	\end{itemize}
 	
    \ressubheading{NASA Kennedy Space Center}{Cape Canaveral, Florida}{Software Engineering Intern}{Jun. 2019 - Aug. 2019}
    \vspace{-0.5\baselineskip}
    \begin{itemize}
        \resitem{Benchmarked and developed class A, safety-critical, human-rated spaceflight ground control software for the Artemis lunar exploration missions}{(C++, Java, Agile)}
    \end{itemize}
        
    \ressubheading{ASU Complex Systems Research Group}{Tempe, Arizona}{Mathematics Research Assistant}{Oct. 2018 - Jun. 2019}
    \vspace{-0.5\baselineskip}
    \begin{itemize}
        \resitem{Analysis and modelling of alarm signal propagation in ants}{(Python, R, Diff. Eq.)}
    \end{itemize}
    
    \ressubheading{Sandia National Laboratories}{Albuquerque, New Mexico}{Quantum Computation Research Intern}{Jun. 2016 - Sep. 2018}
    \vspace{-0.5\baselineskip}
    \begin{itemize}
        \resitem{Created distributed high-performance software for benchmarking \& characterizing ion-trap quantum computers via gradient-based optimization}{(Python, numpy, SLURM)}
    \end{itemize}
    }
    
    \ressection{Publications \& Presentations}{
    \ressubheading{Curiosity in Path-Planning: Synthesizing Path-Planners for Efficient Exploration}{Apr. 15th, 2021}{ICRA "Towards Curious Robots" Workshop}{Virtual}
    \ressubheading{Meta-Learning for Planning: Automatic Synthesis of Sampling-Based Path Planners}{Mar. 26th, 2021}{ICLR Learning-to-Learn Workshop}{Virtual}
    \ressubheading{Qurry, a Quantum Programming Language}{Feb. 2019}{FOSDEM Quantum Computing Development Workshop}{Brussels, Belgium}
    }
    
    \ressection{Projects}{
    \ressubheading{ASU/NASA JPL DORA CubeSat}{Aug., 2020-May 2020}{Ground Software Engineering Student Lead}{Tempe, AZ}
    
    \vspace{-0.5\baselineskip}
    \begin{itemize}
        \resitem{Led development of robust ground station software for the DORA satellite, including radio communications, integration testing, and real-time user interface}{(Rust, Python, KubOS)}
    \end{itemize}
    }
    
 	\vspace{1.0\baselineskip}
\ressection{Skills}{
\begin{description}
    \item[Programming Languages:] 
        Python, C++, Rust, Java, C, x86\_64 Assembly, Clojure (LISPs), Haskell \ldots
 	\vspace{-0.5\baselineskip}
    \item[Technologies:]
        pytorch, tensorflow, numpy, pandas, nltk, plotly, seaborn, matplotlib, Django, neo4j, postgres, linux, AWS, s3, kubernetes, Docker, git, Agile, \LaTeX
\end{description}
}
\end{document}
