\documentclass[letterpaper,11pt]{article}
\usepackage[margin=0.25in]{geometry}
\usepackage{helvet}
\usepackage{setspace}
\renewcommand{\baselinestretch}{1.0} 

\setlength\itemsep{1em}
% Copied from old resume. TODO: Revise
\newcommand{\resitem}[1]{\item #1 \vspace{-3pt}}
\newcommand{\resheading}[1]{
    {\large \textbf{#1}}
    \sectionline
}
\newcommand{\colfill}{@{\extracolsep{\fill}}}
\newcommand{\ressubheading}[4]{
\begin{tabular*}{6.5in}{l\colfill r}
        \textbf{#1} & #2 \\
		\textit{#3} & \textit{#4} \\
\end{tabular*}\vspace{-6pt}}

% Section!
\newcommand{\ressection}[2]
{
\vspace{0.1in}
\begin{minipage}[t]{0.05\textwidth}
    \textbf{#1}
\end{minipage}
\hspace{0.05\textwidth}
\begin{minipage}[t]{0.9\textwidth}
    #2
\end{minipage}
}


\begin{document}
\begin{center}
    \textbf{\Large Lucas Saldyt}
\end{center}
\vspace{-1.5\baselineskip}
\begin{center}
    lucassaldyt@gmail.com $\bullet$ 505-506-1245 $\bullet$ https://github.com/LSaldyt
\end{center}
\vspace{-1.5\baselineskip}
\rule{1.0\textwidth}{1.5pt}
\vspace{-.5\baselineskip}
\newline
    \ressection{Education}{
    \ressubheading{Arizona State University: Barrett, The Honors College}{Tempe, Arizona}{Bachelor of Science in Computer Science, GPA: 3.7}{Sep. 2017 - May 2021}
    }
    \newline
    \ressection{Experience}
    { 
    \ressubheading{National Aeronautics and Space Administration}{Cape Canaveral, Florida}{Software Engineering Intern}{Jun. 2019 - Aug. 2019}
        \begin{itemize}
            \setlength\itemsep{0em}
            \resitem{Worked on class A, safety-critical, human rated spaceflight ground control software by participating in the full software development lifecycle and using agile processes}
             \resitem{Created, benchmarked, and optimized verification/validation software for launch control tests}
            \resitem{Independently prototyped original display profile saving system for launch control engineers}
        \end{itemize}
        \vspace{0.1in}
        \ressubheading{Sandia National Laboratories (Dr. Erik Nielsen)}{Albuquerque, New Mexico}{Quantum Computation Intern}{Jun. 2015 - Sep. 2018}
        \begin{itemize}
            \setlength\itemsep{0em}
            \resitem{Developed high-fidelity quantum benchmarking (Gate Set Tomography) software}
            \resitem{Created distributed high-performance simulation, verification, and data analysis software}
            \resitem{Assisted in publishing papers in quantum benchmarking}
        \end{itemize}
        \vspace{0.1in}
         \ressubheading{Los Alamos National Laboratories (Dr. Scott Pakin)}{Albuquerque, New Mexico}{Quantum Computation Shadow}{Apr. 2017}
        \begin{itemize}
            \setlength\itemsep{0em}
            \resitem{Benchmarked the knapsack problem on LANL's DWave annealer and IBM's machines}
        \end{itemize}
        \vspace{0.1in}
        \ressubheading{ASU Complex Systems Research (Dr. Yun Kang)}{Tempe, Arizona}{Mathematics Research Assistant}{Oct. 2018 - Current}
        \begin{itemize}
            \setlength\itemsep{0em}
            \resitem{Unique math/computer modeling and visualization of ant nest choice and alarm propagation}
            % I would pull any publications out into their own separate section, especially if that's a first authorship -JS
            \resitem{Author of a computation biology paper on alarm propagation, published in PNAS}
        \end{itemize}
        \vspace{0.1in}
         \ressubheading{Fulton Undergraduate Research Initiative (Dr. Ajay Bansal)}{Tempe, Arizona}{Machine Learning Researcher}{Sep. 2018 - Jun. 2019}
        \begin{itemize}
            \setlength\itemsep{0em}
            \resitem{Developed Qurry, a quantum programming language}
            \resitem{Machine learning research, focused around Kolmogorov complexity and program learning}
        \end{itemize}
        \vspace{0.1in}
        \ressubheading{The Fluid Analogies Research Group (Dr. Alexandre Linhares)}{Remote (paid)}{Cognitive Science Research Assistant}{Oct. 2016 - Sep. 2018}
        \begin{itemize}
            \setlength\itemsep{0em}
            \resitem{Revitalized of Douglas Hofstadter's ``copycat'' cognitive model}
            \resitem{Statistical analysis/visualization and comparison of various models to human data}
        \end{itemize}
        \vspace{0.1in}
         \ressubheading{Unitary Fund}{Remote (paid)}{Quantum Software Researcher}{Jun. 2018 - Current}
        \begin{itemize}
                \setlength\itemsep{0em}
                \resitem{Prototyping of a quantum programming language, called ``Qurry''}
                \resitem{Presented in Brussels, Belgium at the FOSDEM Quantum Computing Conference}
        \end{itemize}
        }
        \vspace{0.1in}
        \newline
        \ressection{Skills}{
\begin{description}
    \item[Programming Languages:] 
        Python, C++, Java, Bash, Clojure (LISPs), Haskell, C, MATLAB, R, Fortran
    \item[Applications:]
        Vim, \LaTeX, Git, MPI, Supercomputing (Slurm), Jupyter Notebook, Autodesk Design 
    \item[Operating Systems:]
        Linux, MacOS X, Windows
    \item[Natural Languages:] 
        English, Ukranian, Spanish
\end{description}
}
\end{document}
